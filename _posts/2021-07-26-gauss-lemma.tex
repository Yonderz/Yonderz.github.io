---
title: Gauss's lemma
mathjax: true
tags: number-theory
---
\documentclass{article}
%preamble
\usepackage{amsmath,amsthm,amssymb}
\usepackage[colorlinks=true]{hyperref}
\newtheorem*{thm}{Gauss's lemma}
\newtheorem*{cor}{Corollary}
%\newtheorem{claim}[lem]{Claim}
\theoremstyle{definition}\newtheorem{definition}{Definition}

\begin{document}
	\begin{thm}
		Let p be an odd prime, q be an integer coprime to p. Take the least residues of $Q=\{q, 2q,\cdots,\frac{p-1}{2} q \}$, i.e. reduce them to integers in $[0, p-1]$. Let u be the number of members in this set that are greater than p/2. Then
		$$
		(\frac{q}{p})=-1^u
		$$
	\end{thm}
	
	\begin{proof}
		Q has exactly $\frac{p-1}{2}$ elements since $(p, q)=1$ and $p$ is a prime.We can rewrite Q to integers in $[-\frac{p-1}{2},\frac{p-1}{2}]$ by rewriting all $[o_i]_p : o_i>\frac{p-1}{2}$ into  $[-u_i]_p : u_i=p-o_i \land u_i\leq \frac{p-1}{2}$.
		\begin{align*}
		&\implies \prod_{t=1}^{\frac{p-1}{2}}tq\equiv q^{\frac{p-1}{2}}\cdot {\frac{p-1}{2}}!\equiv{\frac{p-1}{2}}! \cdot (-1)^u \mod p \\
		&\implies (\frac{q}{p}) \equiv q^{\frac{p-1}{2}} \equiv (-1)^u \mod p \tag*{(By Euler's Criterion)}
		\end{align*}
	\end{proof}
	
	\begin{cor}
		Let p be an odd prime. Then
		\begin{equation*}
			(\frac{2}{p})=
				\begin{cases}
					1 & \text{if } p\equiv1 \lor 7 \mod 8 \\
					-1 & \text{if } p\equiv3 \lor 5 \mod 8
				\end{cases}
		\end{equation*}	
	\end{cor}
	
	\begin{proof}
		Because $2\cdot\frac{p-1}{2}=p-1<p$, we only need to find out the cut-off point 
		$$
		n:2n\leq\frac{p-1}{2} \land 2(n+1)>\frac{p-1}{2} \implies n=[\frac{p-1}{4}] \implies u=\frac{p-1}{2}-n=\lceil \frac{p-1}{4} \rceil
		$$
		\begin{enumerate}
			\item $p=8k+1 \implies u=\lceil \frac{8k}{4} \rceil=2k \implies (\frac{p-1}{2})=1$
			\item $p=8k+3 \implies u=\lceil \frac{8k+2}{4} \rceil=2k+1 \implies (\frac{p-1}{2})=-1$
			\item $p=8k+5 \implies u=\lceil \frac{8k+4}{4} \rceil=2k+1 \implies (\frac{p-1}{2})=-1$
			\item $p=8k+7 \implies u=\lceil \frac{8k+6}{4} \rceil=2k+2 \implies (\frac{p-1}{2})=1$
		\end{enumerate}
	\end{proof}
		
\end{document}