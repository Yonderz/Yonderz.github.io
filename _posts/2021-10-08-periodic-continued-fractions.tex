---
title: Periodic Continued Fractions
mathjax: true
tags: number-theory
excerpt:
---
\documentclass{article}
%preamble
\usepackage{parskip}
\usepackage{amsmath,amsthm,amssymb}
\usepackage[colorlinks=true]{hyperref}
\newtheorem*{prop}{Proposition}
%\newtheorem{lem}{Lemma}
%\newtheorem{claim}[lem]{Claim}
\theoremstyle{definition}\newtheorem{definition}{Definition}

\begin{document}
	\begin{prop}
		Suppose that $\alpha$ has an eventually periodic continued fraction expansion. Then $\alpha$ is a quadratic irrational.
	\end{prop}
	\begin{proof}
		We first show this when $\alpha$ has a periodic continued fraction expansion. We then have a $d$ such that
		$$
		\alpha = a _ { 0 } + \frac { 1 } { a _ { 1 } + \frac { 1 } { a _ { 2 } + \cdots + \frac { 1 } { a _ { d - 1 } + \frac { 1 } { \alpha } } } }
		$$
		Since $\alpha_0,\alpha_1, \cdots, \alpha_{d-1}$ are all integers
		$$
		\alpha=\frac{x\alpha+y}{z\alpha+w} \implies z\alpha^2+(w-x)\alpha-y=0
		$$
		Since $\alpha$ is irrational, $z \neq 0$. Thus, $\alpha$ is a quadratic irrational.
		
		If $\alpha=[\alpha_0,\alpha_1,\ldots,\alpha_m,\alpha_{m+1},\ldots,\alpha_{m+d-1},\alpha_{m+d},\ldots]$, then 
		$$
		\beta=\frac{1}{\frac{1}{\frac{1}{\alpha-\alpha_{0}}-\alpha_{1}}-\cdots-\alpha_{m-2}}-\alpha_{m-1}
		$$
		Clearly $\beta$ has a periodic continued fraction expansion. So it is quadratic irrational.
		
		Note,
		\begin{align*}
			\beta=\frac{x\alpha+y}{z\alpha+w} &\implies a\left(\frac{x\alpha+y}{z\alpha+w}\right)^2+b\left(\frac{x\alpha+y}{z\alpha+w}\right)+c=0 \\
			&\implies a'\alpha^2+b'\alpha+c'=0
		\end{align*}
		
		Since $\alpha$ is irrational, $a' \neq 0$. Thus, $\alpha$ is a quadratic irrational.
		
	\end{proof}
			
\end{document}