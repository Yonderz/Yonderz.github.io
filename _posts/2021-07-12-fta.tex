---
title: Fundamental Theorem of Algebra
mathjax: true
tags: complex-analysis
published: true
---

\documentclass{article}
%preamble
\usepackage{amsmath,amsthm,amssymb}
\usepackage[colorlinks=true]{hyperref}
\newtheorem*{thm}{Fundamental Theorem of Algebra}
%{name of the theorem}
%\newtheorem{lem}{Lemma}
%\newtheorem{claim}[lem]{Claim}
\theoremstyle{definition}\newtheorem{definition}{Definition}

\begin{document}
\begin{thm}
	
	Every polynomial of degree greater than zero with complex coefficients has at least one zero.
\end{thm}	

\begin{proof}
	Assume that $p(z)=a_nz^n+a_{n-1}z^{n-1}+\cdots+a_0=0$ has no solutions. Then $1/p(z)$ is entire, i.e. it is differentiable on $\mathbb{C}$. Therefore $1/p(z)$ is also continuous. Also notes, as $|z| \rightarrow \infty$, $|1/p(z)| \rightarrow 0$
	
	Thus, $$\exists M_R>0 :(  \exists R>0:\forall z \in B_R=\{z\in \mathbb{C}: |z|\leq R\}, |1/p(z)| \leq M_R)$$
	

	$1/p(z)$ is entire and bounded so it is a constant. But $p(z)$ is not a constant. Contradiction, thus it must has at least one zero.
\end{proof}			
			
\end{document}