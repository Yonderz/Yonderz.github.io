---
title: First Isomorphism Theorem for Groups
mathjax: true
tags: group-theory
excerpt:
---
\documentclass{article}
%preamble
\usepackage{parskip}
\usepackage{amsmath,amsthm,amssymb}
\usepackage[colorlinks=true]{hyperref}
\newtheorem*{thm}{First Isomorphism Theorem for Groups}
%\newtheorem{lem}{Lemma}
%\newtheorem{claim}[lem]{Claim}
\theoremstyle{definition}\newtheorem{definition}{Definition}

\begin{document}
	\begin{thm}
		If $\phi : G \rightarrow H$ is a homomorphism then
$$
G / \operatorname { ker } ( \phi ) \cong \operatorname { im } ( \phi )
$$
	\end{thm}
	\begin{proof}
		Define a mapping $f:G / \operatorname { ker } ( \phi ) \rightarrow \operatorname { im } ( \phi )$ by $f(a\operatorname { ker } ( \phi ))= \phi  (a)$
		
		The map is well-defined since if $a\operatorname { ker } ( \phi )=b\operatorname { ker } ( \phi )$, then $a^{-1}b \in \operatorname { ker } ( \phi )$ and
		$$
		f(a\operatorname { ker } ( \phi ))= \phi  (a)\cdot e_H=\phi  (a)\cdot \phi  (a^{-1}b)=\phi(b)=f(b\operatorname { ker } ( \phi ))
		$$
		
		Let $h$ be an arbitrary element of $\operatorname { im } ( \phi )$, then $\exists g \in G : \phi(g)=h$. Since $f(g\operatorname { ker } ( \phi ))=\phi(g)=h$, $f:G / \operatorname { ker } ( \phi ) \rightarrow \operatorname { im } ( \phi )$ is surjective. It is also injective because $f(a\operatorname { ker } ( \phi ))=f(b\operatorname { ker } ( \phi )) \implies \phi(a)=\phi(b) \implies \phi(a)\phi(b^{-1})=e_H \implies ab^{-1}\in \operatorname { ker } ( \phi ) \implies a\operatorname { ker } ( \phi )=b\operatorname { ker } ( \phi )$
		
		$f:G / \operatorname { ker } ( \phi ) \rightarrow \operatorname { im } ( \phi )$ is a homomorphism since 
		$$
		f(a\operatorname { ker } ( \phi ) \cdot b\operatorname { ker } ( \phi ))=f((ab)\operatorname { ker } ( \phi ))=\phi(ab)=\phi(a)\phi(b)=f(a\operatorname { ker } ( \phi ) )f( b\operatorname { ker } ( \phi ))
		$$
	\end{proof}		
\end{document}