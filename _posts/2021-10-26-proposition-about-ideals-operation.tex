---
title: Proposition about Ideals Operation
mathjax: true
tags: algebra
excerpt: <blockquote class="statement"> <p><strong>Proposition</strong> If <span class="inline_math">\( I + J = \langle 1 \rangle \)</span>, then <span class="inline_math">\( I \cap J = I J \)</span></p> </blockquote>
---
\documentclass{article}
%preamble
\usepackage{parskip}
\usepackage{amsmath,amsthm,amssymb}
\usepackage[colorlinks=true]{hyperref}
\newtheorem*{prop}{Proposition}
%\newtheorem{lem}{Lemma}
%\newtheorem{claim}[lem]{Claim}
\theoremstyle{definition}\newtheorem{definition}{Definition}

\begin{document}
	\begin{prop}
		If $I + J = \langle 1 \rangle$, then $I \cap J = I J$
	\end{prop}
	\begin{proof}\
		\begin{itemize}
			\item [$\subseteq:$] $\forall t \in I \cap J , t \in I \land t \in J$. $I + J = \langle 1 \rangle \implies \exists i_1 , j_1 \in I, J : i_1+j_1=1$. Note, $t=i_1t+j_1t \in IJ$. So, $I \cap J \subseteq I J$ 
			\item [$\supseteq:$] $\forall t \in I J, \exists i_k , j_k \in I, J : t= \sum {i_kj_k}$. Note, for all $k$, $i_k , j_k \in I, R \implies i_kj_k \in I $. Since $I$ is closed under addition, $t= \sum {i_kj_k} \in I $. 
			
			Similarly, $t \in J$. Thus, $t \in I \cap J$ and $I \cap J \supseteq I J$
		\end{itemize}
		Overall, $I \cap J = I J$
	\end{proof}		
\end{document}