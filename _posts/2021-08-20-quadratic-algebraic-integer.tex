---
title: Quadratic algebraic integer
mathjax: true
tags: number-theory
excerpt: <blockquote class="statement"> <p><strong>Proposition</strong> If <span class="inline_math">\( \alpha \)</span> is an algebraic integer of degree two, then <span class="inline_math">\( \mathbb Z [\alpha] \)</span> is equal to the set of complex numbers of the form <span class="inline_math">\( x + y\alpha \)</span>, where x and y are integers.<span class="intersentencespace"></span></p> </blockquote>
---
\documentclass{article}
%preamble
\usepackage{parskip}
\usepackage{amsmath,amsthm,amssymb}
\usepackage[colorlinks=true]{hyperref}
\newtheorem*{proposition}{Proposition}
%\newtheorem{lem}{Lemma}
%\newtheorem{claim}[lem]{Claim}
\theoremstyle{definition}\newtheorem{definition}{Definition}

\begin{document}
\begin{definition}
	An element $\alpha$ of $\mathbb C$ is an algebraic integer of degree two (alternatively, a quadratic algebraic integer) if there exists a polynomial of the form $P (X ) = X^ 2 + aX + b$ with a, b integers roots such that $P (X )$ has
no rational roots and $P (\alpha) = 0$.
\end{definition}
\
\begin{proposition}
	 If $\alpha$ is an algebraic integer of degree two, then $\mathbb Z [\alpha]$ is equal to the set of complex numbers of the form $x + y\alpha$, where x and y are integers.
\end{proposition}
\begin{proof}
	Let set S be the set of complex numbers of the form $x + y\alpha$, where x and y are integers.
	
	Then it is obvious that S is a subset of $\mathbb Z [\alpha]$.
	
	$\forall x,y,z \in S, xy=x_2 y_2 \alpha^2 + (x_1y_2+x_2y_1) \alpha + x_1 y_1= (x_2 y_2)\cdot(-a\alpha-b)+ (x_1y_2+x_2y_1) \alpha + x_1 y_1 = yx \implies xy \in S$. Note $(x+y)z=xz+yz$. So S is a subring of $\mathbb C$. $\alpha \in S \implies S \supseteq \mathbb Z [\alpha]$ 
	
	$S \subseteq \mathbb Z [\alpha] \, \land \, S \supseteq \mathbb Z [\alpha] \implies S=\mathbb Z [\alpha]$
\end{proof}	
			
\end{document}
