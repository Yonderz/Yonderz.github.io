---
title: Fermat's Two-Square Theorem
mathjax: true
tags: number-theory
excerpt: Every prime congruent to $1 \bmod 4$ is the sum of two squares.
---
\documentclass{article}
%preamble
\usepackage{parskip}
\usepackage{amsmath,amsthm,amssymb}
\usepackage[colorlinks=true]{hyperref}
\newtheorem*{thm}{Fermat's Two-Square Theorem}
\newtheorem{prop}{Proposition}
%\newtheorem{claim}[lem]{Claim}
\theoremstyle{definition}\newtheorem{definition}{Definition}

\begin{document}
	\begin{prop}[Diophantus identity] \label{Diophantus identity}
		The product of two numbers, each of which is a sum of two squares, is itself a sum of two squares.
	\end{prop}
	\begin{proof}
		\begin{align*}
			(a^2+b^2)\cdot (c^2+d^2) &= a^2c^2+b^2d^2+a^2d^2+b^2c^2 \\
			&= a^2c^2+2abcd+b^2d^2+a^2d^2-2abcd+b^2c^2 \\
			&= (ac+bd)^2+(ad-bc)^2
		\end{align*}
	\end{proof}

	\begin{prop} \label{Red}
		 Suppose we have $a^2 + b^2 = rp$, with a, b integers, $ab\neq0$, and $1 < r < p$. Then there exists $1 \leq r' < r$ and integers x, y such that $x^2 + y^2 = r'p$.
	\end{prop}
	\begin{proof}
		Find $u\equiv a \bmod r , v\equiv b \bmod r$ where $u,v \in [\frac{-r}{2},\frac{r}{2}]$. 
		
		$u^2+v^2 \equiv rp \equiv 0 \mod r \implies u^2+v^2=r' r$
				
		$0<u^2+v^2 \leq \frac{r^2}{4} + \frac{r^2}{4}=\frac{r^2}{2} \implies 1 \leq r'\leq \frac{r}{2} < r$
		
		Based on \hyperref[Diophantus identity]{Diophantus identity}
		\begin{align*}
			r^2r'p &= (a^2+b^2)\cdot(u^2+v^2)\\
					&= (au+bv)^2 + (av-bu)^2
		\end{align*} 
		$au+bv \equiv a^2+b^2 \equiv 0 \mod r \implies r^2 \mid (au+bv)^2$
		
		$av-bu \equiv ab-ba \equiv 0 \mod r \implies r^2 \mid (av-bu)^2$
		
		Thus
		\begin{equation*}
			r'p=\frac{r^2r'p}{r^2}=(\frac{au+bv}{r})^2 + (\frac{av-bu}{r})^2
		\end{equation*}
	\end{proof}

	\begin{thm}
		Every prime congruent to $1 \bmod 4$ is the sum of two squares.
	\end{thm}
	\begin{proof}
		Based on Euler's Criterion, $[-1]_p$ is a quadratic residue. 
		
		$\implies \exists d \in [-\frac{p-1}{2},\frac{p-1}{2}] : p \mid (d^2+1)$ 
		
		$0<d^2+1<\frac{p^2}{4}+1 \leq \frac{p^2}{2} \implies \exists r \in (0, p) : d^2+1^2=rp$
		
		If $r=1$, then $p=d^2+1^2$.
		
		If $r>1$, based on \hyperref[Red]{Proposition 2} we can keep reducing $rp$ to $r'p$ which is still sum of two squares until $r'$ is reduced to 1. Then $r'p=p=x^2+y^2$ 
	\end{proof}		
\end{document}